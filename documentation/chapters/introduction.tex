\documentclass[documentation.tex]{subfiles}
\begin{document}
	\begin{text}\par
WebFrame++ е проект, разработен в полза, както на големи и средни, така и на малки компании, които се занимават с разработката на интернет приложения. Библиотеката е подходяща за употреба в училища и университети в България, тъй като там се изучава програмният език C++,  което ще мотивира още повече хора да се занимават професионално с езика. Също така библиотеката подпомага разрастването на C++ обществото по целия свят.\par
Проблемът на всеки разработчик на интернет приложения на C++ е, че трябва да използва стари и бавни методи като CGI и FastCGI, които са значително по-неефективни от наложените в момента технологии в тази област. Разработчиците имат и опцията да използват socket, които е метод от езика C, който работи на значително по-ниско ниво и не използва модернизациите във съвременните версии на C++ - C++20, C++17, дори C++14 и C++11, което го прави труден за използване от програмистите.\par
От друга страна езикът C++ е широко разпространен в сфери като търсачки, сложни алгоритми, изкуствен интелект, големи данни и други. Това означава, че благодарение на WebFrame++ разработчиците ще мога да съчетават всички тези тематики с уеб приложения и REST APIs за мобилните си приложения, които също може да бъдат имплементирани на C++.\par
От гледна точка на бизнеса библиотеката ще намали очакваното ниво на знания за работниците, тъй като няма да има нужда те да мога да четат/разбират и пишат на езици, различни от C++ (за сървърната част), стандартните HTML, CSS и JavaScript (за изгледа на страниците) и евентуално JSON, ако се налага в съответната компания.\par
Наличието на повече кандидати с достатъчни знание, съответно, води до по-бързо разрастване на компанията и по-голяма конкуренция между кандидатите.
    \end{text}
    \newpage
\end{document} 
